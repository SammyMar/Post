% Options for packages loaded elsewhere
\PassOptionsToPackage{unicode}{hyperref}
\PassOptionsToPackage{hyphens}{url}
%
\documentclass[
]{article}
\usepackage{amsmath,amssymb}
\usepackage{lmodern}
\usepackage{iftex}
\ifPDFTeX
  \usepackage[T1]{fontenc}
  \usepackage[utf8]{inputenc}
  \usepackage{textcomp} % provide euro and other symbols
\else % if luatex or xetex
  \usepackage{unicode-math}
  \defaultfontfeatures{Scale=MatchLowercase}
  \defaultfontfeatures[\rmfamily]{Ligatures=TeX,Scale=1}
\fi
% Use upquote if available, for straight quotes in verbatim environments
\IfFileExists{upquote.sty}{\usepackage{upquote}}{}
\IfFileExists{microtype.sty}{% use microtype if available
  \usepackage[]{microtype}
  \UseMicrotypeSet[protrusion]{basicmath} % disable protrusion for tt fonts
}{}
\makeatletter
\@ifundefined{KOMAClassName}{% if non-KOMA class
  \IfFileExists{parskip.sty}{%
    \usepackage{parskip}
  }{% else
    \setlength{\parindent}{0pt}
    \setlength{\parskip}{6pt plus 2pt minus 1pt}}
}{% if KOMA class
  \KOMAoptions{parskip=half}}
\makeatother
\usepackage{xcolor}
\usepackage[margin=1in]{geometry}
\usepackage{graphicx}
\makeatletter
\def\maxwidth{\ifdim\Gin@nat@width>\linewidth\linewidth\else\Gin@nat@width\fi}
\def\maxheight{\ifdim\Gin@nat@height>\textheight\textheight\else\Gin@nat@height\fi}
\makeatother
% Scale images if necessary, so that they will not overflow the page
% margins by default, and it is still possible to overwrite the defaults
% using explicit options in \includegraphics[width, height, ...]{}
\setkeys{Gin}{width=\maxwidth,height=\maxheight,keepaspectratio}
% Set default figure placement to htbp
\makeatletter
\def\fps@figure{htbp}
\makeatother
\setlength{\emergencystretch}{3em} % prevent overfull lines
\providecommand{\tightlist}{%
  \setlength{\itemsep}{0pt}\setlength{\parskip}{0pt}}
\setcounter{secnumdepth}{-\maxdimen} % remove section numbering
\ifLuaTeX
  \usepackage{selnolig}  % disable illegal ligatures
\fi
\IfFileExists{bookmark.sty}{\usepackage{bookmark}}{\usepackage{hyperref}}
\IfFileExists{xurl.sty}{\usepackage{xurl}}{} % add URL line breaks if available
\urlstyle{same} % disable monospaced font for URLs
\hypersetup{
  pdftitle={Testes de hipóteses},
  pdfauthor={Samuel Martins de Medeiros},
  hidelinks,
  pdfcreator={LaTeX via pandoc}}

\title{Testes de hipóteses}
\author{Samuel Martins de Medeiros}
\date{}

\begin{document}
\maketitle

\hypertarget{introduuxe7uxe3o}{%
\subsubsection{Introdução}\label{introduuxe7uxe3o}}

De maneira geral, existem duas grandes áreas na Estatística: a estimação
de parâmetros e o teste de hipóteses. Em particular, testar hipóteses é
testar uma hipótese nula contra uma hipótese alternativa, em que na
hipótese nula (\(H_0\)) se faz uma afirmação a respeito de um parâmetro
e na hipótese alternativa (\(H_a\) ou ainda \(H_1\)) é informado o
contrário dessa afirmação.

Para deixar essa ideia um pouco mais clara, tomemos um exemplo: suponha
que gostaríamos de saber qual marca, A ou B, de um determinado tipo de
blusa dura mais tempo. Conseguimos reescrever esse problema em formato
de hipóteses, a saber:

Além disso, utilizamos teste de hipóteses para validar um espaço de
possíveis valores para o parâmetro sob investigação (a média, por
exemplo). Assim como no estudo de estimação, assumimos que é possível
obter uma amostra aleatória X1,\ldots,Xn de uma distribuição f(.;θ).
Portanto, uma hipótese estatística seria uma hipótese a respeito da
distribuição da população.

Antes de mais nada, precisamos definir o que é uma estatística de teste.
Estatística de teste pode ser descrita como a regra de rejeição para uma
hipótese. Tomemos mais um exemplo: seja X1,\ldots,Xn uma amostra
aleatória de função densidade f(x;θ) e a hipótese nula a ser testada H0:
θ \textless{} 17. Um possível teste seria rejeitar \(H_0\) se e somente
se . Formalmente, chamamos esse conjunto de possíveis resultados, para o
qual se rejeita \(H_0\), de região crítica.

Para entender melhor o conceito de região crítica (ou \textbf{região de
rejeição}), vamos considerar outro exemplo: seja \(X\) uma variável
aleatória de uma distribuição dependente do parâmetro θ. Considere,
agora, \(T = r(X)\) uma estatística e \(R\) um subconjunto de valores
reais. Suponha um procedimento de teste de hipóteses da forma ``rejeite
\(H_0\) se \(T\) pertence à R''. Então, \(T\) é uma estatística de teste
e \(R\) é a \textbf{região de rejeição} do teste. Um teste pode ser
tanto aleatório quanto não aleatório. O exemplo anterior, por exemplo, é
um ótimo exemplo de teste não aleatório. Já um teste aleatório poderia
ser ``jogue uma moeda para o alto, caso cara rejeite a hipótese nula''.

Um teste pode ser tanto randômico quanto não randômico. O acima citado é
um ótimo exemplo de teste não randomizado, uma possível alternativa para
randômico seria: jogue uma moeda para o alto, caso cara rejeite a
hipótese nula.

Exemplo: Por último, vamos supor que \(X = (X1,...,Xn)\) é uma amostra
aleatória de uma distribuição Normal com média μ desconhecida e
variância σ² conhecida. Queremos testar as hipóteses:

É sensato rejeitar \(H_0\) se for muito discrepante de . Por exemplo,
nós poderíamos escolher um número \(c\) qualquer e rejeitar a hipótese
nula se a distância de para for maior do que c.~Assim, a estatística do
teste seria dada por , a região de não rejeição seria definida como e
seu complementar sendo, portanto, a região de rejeição do teste.

\hypertarget{funuxe7uxe3o-poder-e-tipos-de-erros}{%
\paragraph{Função Poder e Tipos de
Erros}\label{funuxe7uxe3o-poder-e-tipos-de-erros}}

Para cada teste aplicado sobre uma amostra obtida de uma distribuição em
que θ ∈ Θ, teremos uma função poder associada. ● Função Poder: suponha
um procedimento de teste δ. A função π(θ\textbar δ) é chamada função
poder do teste δ. Se \(S\) denota a região crítica de δ, então a função
poder é determinada pela relação: , para todo θ ∈ Θ. Se δ é descrito em
função da estatística de teste \(T\) e da região de rejeição \(R\),
então , para todo θ ∈ Θ. Como a função poder é definida para todo espaço
paramétrico, podemos escrever o teste de hipóteses em função do espaço
paramétrico, isto é, supondo a hipótese nula em que θ é igual a 0 e a
hipótese alternativa em que θ é igual a 1. É importante dizer que o
poder nada mais é do que a probabilidade de rejeitar \(H_0\) dados os
possíveis valores de θ. Exemplo: considere o último exemplo da
Introdução, em que lidamos com uma distribuição Normal com média
desconhecida e variância conhecida e região crítica \(RC = [c,∞)\).
Logo, a distribuição de também é uma normal de média μ, porém com
variância σ²/n.~E portanto, a função poder é dada por:

Como dito inicialmente, testamos se a hipótese nula é falsa e, portanto,
se a hipótese alternativa é verdadeira, ou vice-versa. Nesse contexto,
dois tipos de erros podem ser cometidos: + Erro do Tipo I: rejeitar a
hipótese nula quando a mesma é verdadeira. + Erro do Tipo II: aceitar a
hipótese nula quando a mesma é falsa.

\hypertarget{hipuxf3tese-simples-versus-hipuxf3tese-simples}{%
\subsubsection{Hipótese simples versus Hipótese
simples}\label{hipuxf3tese-simples-versus-hipuxf3tese-simples}}

As hipóteses de um teste podem ser da forma simples ou composta. Em uma
simples, é especificada completamente a distribuição de um parâmetro θ
(\(H_0: µ = 0\), como exemplo). Por outro lado, uma hipótese composta é
aquela cuja distribuição (\(H_0:µ >50\)) não é especificada
completamente. A discussão acerca de hipótese simples versus hipótese
composta não é muito vista na prática, porém serve como ótima introdução
ao tema.

\hypertarget{testes-de-razuxe3o-de-verossimilhanuxe7a-simples}{%
\paragraph{Testes de razão de verossimilhança
simples}\label{testes-de-razuxe3o-de-verossimilhanuxe7a-simples}}

Suponha X1,\ldots,Xn sendo uma amostra aleatória de uma distribuição
\(fº(.)\) ou \(f¹(.)\). Um teste de \(H_0: Xi ~ fº(.)\)
vs.~\(H_1: Xi ~ f¹(.)\) é um teste da razão de verossimilhança se é
definido como: onde \(k\) é uma constante não negativa e
\(L(x_1,…,x_n)\) é a função de verossimilhança associada à função de
densidade \(f(.)\). Rejeitamos a hipótese nula para um valor de lambda
pequeno, pois, seguindo a linha de raciocínio da razão das funções,
\(L¹\) é maior que \(Lº\). Dessa forma, há mais indícios de que a
amostra venha de uma população com distribuição \(f¹(.)\) em vez de uma
\(fº(.)\). Exemplo: seja \(X_1,…,X_n\) uma amostra aleatória de uma
distribuição \(N(μ,1)\) na qual queremos testar \(H_0: μ = 0\)
\times \(H_1: μ = 1\). Nesse caso, o teste da razão de verossimilhança
pode ser dado por:

Ou seja, rejeitamos \(H_0\) se o somatório de \(x\) for maior do que
alguma constante \(k*\).

Ou seja, rejeitamos H0 para um somatório de \(x\) maior que alguma
constante k*.

\hypertarget{testes-mais-poderosos}{%
\paragraph{Testes Mais Poderosos}\label{testes-mais-poderosos}}

Antes de falar sobre os testes mais poderosos, uma definição deve ser
esclarecida: o tamanho do teste. Vamos admitir um teste δ cuja hipótese
nula seja \(H_0: θ ∈ Θº\), em que Θº ⊂ Θ (ou seja, Θº é um subconjunto
do espaço paramétrico Θ). Assim, o tamanho do teste é definido como .
Esclarecida essa definição, daremos prosseguimento ao assunto. Assim
como já comentado, queremos um teste δ em que π(θº) = P{[}Rejeitar H0
\textbar{} H0 verdadeiro{]} seja a menor possível e que π(θ¹) =
P{[}Rejeitar H0 \textbar{} H0 falsa{]} seja a maior possível. Em um
mundo perfeito, π(θ¹) = 1 e π(θº) = 0, isto é, quando os erros do tipo I
e II são minimizados simultâneamente. Entretanto, na prática, uma das
metodologias aplicadas de forma a definir o melhor teste possível é
minimizar o erro do tipo II fixando o erro do tipo I.

\textbf{Teste Mais Poderoso}: Um teste δ* em que \(H_0: θ = θº\) contra
\(H_1: θ = θ¹\) é definido como teste mais poderoso de tamanho α(0
\textless{} α \textless{} 1) se e somente se: i. π(θº\textbar δ\emph{) =
α; ii. π(θ¹\textbar δ}) \textgreater{} π(θ¹\textbar δ), para qualquer
outro teste δ onde π(θº\textbar δ) \textless{} α. Ou seja, podemos
considerar um teste δ* como sendo o teste mais poderoso se, para
qualquer outro teste de tamanho α ou menor do que α, ele possuir o maior
poder.

O lema (ou método) a seguir é muito útil para encontrar testes mais
poderosos.

\begin{itemize}
\item
  \textbf{Lemma Neyman-Pearson:} seja \(X_1,...,X_n\) uma amostra
  aleatória de uma distribuição com densidade f(x;θ), onde θ pode
  assumir os valores θ¹ ou θº e 0 \textless{} α \textless{} 1. Considere
  \(k*\) uma constante positiva e C* um subconjunto do espaço de valores
  para \(X_i\). Assim,
\end{itemize}

e λ \textgreater{} k* se

Então, considerando um teste de hipóteses simples, temos que o teste
para essa região crítica é o teste mais poderoso. Vamos mostrar um
exemplo para melhor compreensão.

Exemplo: seja \(X_1,...,X_n\) uma amostra aleatória de uma distribuição
Bernoulli(θ) e seja o teste \(H_0:θ = θ°\) vs.~\(H_1: θ = θ¹\),
\(θ¹ > θº\). O teste mais poderoso de tamanho α para testar \(H_0\)
contra \(H_1\) é da forma

onde \(k\) e \(γ\) é determinada de maneira que . Agora, se

dado que \(θ¹ > θº\) e \(λ(x)\) é uma função crescente de ,segue que
\(λ(x) > k\) se e somente se \textgreater{} k¹, sendo \(k¹\) uma
constante. Então, o teste mais poderoso de tamanho α é da forma

Ainda, \(k¹\) e \(γ\) são determinados da forma

Observe que o teste mais poderoso de tamanho α é independente de \(θ¹\)
quando \(θ¹ > θº\), e é, portanto, o teste mais poderoso para verificar
se \(θ = θº\) contra \(θ > θº\).

\hypertarget{testes-para-hipuxf3teses-compostas}{%
\subsubsection{Testes para hipóteses
compostas}\label{testes-para-hipuxf3teses-compostas}}

Após todos os conceitos visto até aqui, nosso trabalho passou a ser mais
simples nesta seção, pois apenas generalizaremo-los para os teste de
hipóteses compostos. A princípio, começaremos com o método mais geral
para testar hipóteses, que, geralmente, não é o que fornece resultados
mais precisos, mas é aplicável em todo tipo de situação. Considere
\(X_1,...,X_n\) uma amostra aleatória obtida de uma função de densidade
f(x;θ),θ ∈ Θ, e um teste do tipo \(H_0: θ ∈ Θ°\) contra
\(H_1: θ ∈ Θ¹ = Θ - Θ°\).

\begin{itemize}
\tightlist
\item
  \textbf{Teste de Razão de Verossimilhança Generalizada}: suponha
  \(L(θ;X1,...,Xn)\) a função de verossimilhança para a amostra
  \(X_1,...,X_n\). O teste de razão de verossimilhança generalizada,
  denotado por \(λ\), é definido como:
\end{itemize}

onde λ se torna uma função da amostra definida no intervalo {[}0,1{]}.
Assim como no Teste de Razão de Verossimilhança simples(?), rejeitamos a
\(H_0\) para algum λ° \textgreater{} λ, em que λ° é uma constante
definida no intervalo {[}0,1{]}.

Agora partimos para os Testes Uniformememnte Mais Poderosos, que tem
como definição:

\begin{itemize}
\item
  \textbf{Testes Uniformemente Mais Poderosos (TUMP)}: um teste δ* do
  tipo \(H_0: θ ∈ Θ°\) contra \(H_1: θ ∈ Θ¹ = Θ - Θ°\) é definido como
  TUMP de tamanho α se e somente se

  para todo θ ∈ Θ - Θ° e para qualquer teste δ de tamanho menor ou igual
  a α.
\end{itemize}

\hypertarget{conclusuxe3o}{%
\subsubsection{Conclusão}\label{conclusuxe3o}}

Na literatura, podemos encontrar formas diferentes de testar hipóteses
das vistas neste tutorial, mas elas fogem do escopo deste post e por
isso não foram abordadas. Ainda assim, fomos capazes de aprender alguns
dos métodos para testar hipóteses estatísticas mais utilizados no meio
da graduação e também profissional, além de métodos para achar o melhor
tipo de teste. Espero que o texto tenha sido esclarecedor e de ajuda ao
leitor. Para mais informações ou dúvidas, escreva-nos em :
\href{mailto:comunicacao@observatorioobstetricobr.org}{\nolinkurl{comunicacao@observatorioobstetricobr.org}}

\end{document}
